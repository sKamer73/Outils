%% Generated by Sphinx.
\def\sphinxdocclass{report}
\documentclass[letterpaper,10pt,english]{sphinxmanual}
\ifdefined\pdfpxdimen
   \let\sphinxpxdimen\pdfpxdimen\else\newdimen\sphinxpxdimen
\fi \sphinxpxdimen=.75bp\relax

\PassOptionsToPackage{warn}{textcomp}
\usepackage[utf8]{inputenc}
\ifdefined\DeclareUnicodeCharacter
% support both utf8 and utf8x syntaxes
  \ifdefined\DeclareUnicodeCharacterAsOptional
    \def\sphinxDUC#1{\DeclareUnicodeCharacter{"#1}}
  \else
    \let\sphinxDUC\DeclareUnicodeCharacter
  \fi
  \sphinxDUC{00A0}{\nobreakspace}
  \sphinxDUC{2500}{\sphinxunichar{2500}}
  \sphinxDUC{2502}{\sphinxunichar{2502}}
  \sphinxDUC{2514}{\sphinxunichar{2514}}
  \sphinxDUC{251C}{\sphinxunichar{251C}}
  \sphinxDUC{2572}{\textbackslash}
\fi
\usepackage{cmap}
\usepackage[T1]{fontenc}
\usepackage{amsmath,amssymb,amstext}
\usepackage{babel}



\usepackage{times}
\expandafter\ifx\csname T@LGR\endcsname\relax
\else
% LGR was declared as font encoding
  \substitutefont{LGR}{\rmdefault}{cmr}
  \substitutefont{LGR}{\sfdefault}{cmss}
  \substitutefont{LGR}{\ttdefault}{cmtt}
\fi
\expandafter\ifx\csname T@X2\endcsname\relax
  \expandafter\ifx\csname T@T2A\endcsname\relax
  \else
  % T2A was declared as font encoding
    \substitutefont{T2A}{\rmdefault}{cmr}
    \substitutefont{T2A}{\sfdefault}{cmss}
    \substitutefont{T2A}{\ttdefault}{cmtt}
  \fi
\else
% X2 was declared as font encoding
  \substitutefont{X2}{\rmdefault}{cmr}
  \substitutefont{X2}{\sfdefault}{cmss}
  \substitutefont{X2}{\ttdefault}{cmtt}
\fi


\usepackage[Bjarne]{fncychap}
\usepackage[,numfigreset=1,mathnumfig]{sphinx}

\fvset{fontsize=\small}
\usepackage{geometry}


% Include hyperref last.
\usepackage{hyperref}
% Fix anchor placement for figures with captions.
\usepackage{hypcap}% it must be loaded after hyperref.
% Set up styles of URL: it should be placed after hyperref.
\urlstyle{same}

\usepackage{sphinxmessages}




\title{MathBook}
\date{Nov 27, 2020}
\release{}
\author{Simon Kamerling}
\newcommand{\sphinxlogo}{\vbox{}}
\renewcommand{\releasename}{}
\makeindex
\begin{document}

\pagestyle{empty}
\sphinxmaketitle
\pagestyle{plain}
\sphinxtableofcontents
\pagestyle{normal}
\phantomsection\label{\detokenize{intro::doc}}


This is a small sample book to give you a feel for how book content is
structured.

Check out the content pages bundled with this sample book to get started.


\chapter{Plotter}
\label{\detokenize{Part1/Plot:plotter}}\label{\detokenize{Part1/Plot::doc}}
Nous allons ici plotter via plotly et cufflinks, qui permettent une visualisation rapide et efficace d’un certain nombre de données.

Il est nécessaire de se renseigner sur pandas et les DataFrame pour obtenir les fonctions de manipulation les plus pratiques.

La méthode implémentée ici est de créer un DataFrame , puis d’appeler dessus la méthode .iplot() (implémentée dans cufflinks). Une autre méthode est d’utiliser directement plotly en rajoutant différents layout sur une figure.

Il y a un certain nombre d’options dans iplot(); par exemple:
\begin{itemize}
\item {} 
y = {[}‘mdot(h)’{]} \sphinxhyphen{}\textgreater{} choisit la fonction en y, possibilité de mettre plusieurs clés

\item {} 
secondary\_y \sphinxhyphen{}\textgreater{} pareil mais pour le deuxième axe

\item {} 
kind = ‘box’ \sphinxhyphen{}\textgreater{} type de graphiques. Une courbe simple: ‘scatter’; beaucoup de styles sont implémentés

\end{itemize}

De nombreux paramètres sont accessibles via help(cf.iplot)

\begin{sphinxVerbatim}[commandchars=\\\{\}]
\PYG{k+kn}{import} \PYG{n+nn}{pandas} \PYG{k}{as} \PYG{n+nn}{pd}
\PYG{k+kn}{from} \PYG{n+nn}{ipywidgets} \PYG{k+kn}{import} \PYG{n}{interact}\PYG{p}{,} \PYG{n}{interactive}\PYG{p}{,} \PYG{n}{fixed}\PYG{p}{,} \PYG{n}{interact\PYGZus{}manual}\PYG{p}{,} \PYG{n}{IntSlider}
\PYG{c+c1}{\PYGZsh{} Standard plotly imports}
\PYG{k+kn}{import} \PYG{n+nn}{chart\PYGZus{}studio}\PYG{n+nn}{.}\PYG{n+nn}{plotly} \PYG{k}{as} \PYG{n+nn}{py}
\PYG{k+kn}{import} \PYG{n+nn}{plotly}\PYG{n+nn}{.}\PYG{n+nn}{graph\PYGZus{}objs} \PYG{k}{as} \PYG{n+nn}{go}
\PYG{k+kn}{from} \PYG{n+nn}{plotly}\PYG{n+nn}{.}\PYG{n+nn}{offline} \PYG{k+kn}{import} \PYG{n}{iplot}\PYG{p}{,} \PYG{n}{init\PYGZus{}notebook\PYGZus{}mode}
\PYG{c+c1}{\PYGZsh{} Using plotly + cufflinks in offline mode}
\PYG{k+kn}{import} \PYG{n+nn}{cufflinks} \PYG{k}{as} \PYG{n+nn}{cf}
\PYG{n}{cf}\PYG{o}{.}\PYG{n}{go\PYGZus{}offline}\PYG{p}{(}\PYG{n}{connected}\PYG{o}{=}\PYG{k+kc}{False}\PYG{p}{)}
\PYG{n}{init\PYGZus{}notebook\PYGZus{}mode}\PYG{p}{(}\PYG{n}{connected}\PYG{o}{=}\PYG{k+kc}{False}\PYG{p}{)}
\end{sphinxVerbatim}

\sphinxstylestrong{Exo 1} : Pour n entier naturel non nul, on pose \(H_n = \sum_{k=1}^n \frac{1}{k} \)  (série harmonique).
\begin{enumerate}
\sphinxsetlistlabels{\arabic}{enumi}{enumii}{}{.}%
\item {} 
Montrer que : \(\forall n \in \mathbb{N}, \ln(n + 1) < H_n < 1 + \ln(n)\) et en déduire la limite en \(+\infty\) de \(H_n\).

\item {} 
Pour n entier naturel non nul, on pose \(u_n = H_n − ln(n)\) et \(v_n = H_n − ln(n + 1)\). Montrer que les suites \((u_n)\) et \((v_n)\) convergent vers un réel γ.

\end{enumerate}

\begin{sphinxVerbatim}[commandchars=\\\{\}]
\PYG{k+kn}{from} \PYG{n+nn}{math} \PYG{k+kn}{import} \PYG{n}{log}
\PYG{n}{nMax}\PYG{o}{=}\PYG{l+m+mi}{100}
\PYG{n}{logN}\PYG{o}{=}\PYG{p}{[}\PYG{l+m+mi}{1}\PYG{o}{+}\PYG{n}{log}\PYG{p}{(}\PYG{n}{n}\PYG{p}{)} \PYG{k}{for} \PYG{n}{n} \PYG{o+ow}{in} \PYG{n+nb}{range}\PYG{p}{(}\PYG{l+m+mi}{1}\PYG{p}{,}\PYG{n}{nMax}\PYG{o}{+}\PYG{l+m+mi}{1}\PYG{p}{)}\PYG{p}{]}
\PYG{n}{logNPlus1}\PYG{o}{=}\PYG{p}{[}\PYG{n}{log}\PYG{p}{(}\PYG{n}{n}\PYG{o}{+}\PYG{l+m+mi}{1}\PYG{p}{)} \PYG{k}{for} \PYG{n}{n} \PYG{o+ow}{in} \PYG{n+nb}{range}\PYG{p}{(}\PYG{l+m+mi}{1}\PYG{p}{,}\PYG{n}{nMax}\PYG{o}{+}\PYG{l+m+mi}{1}\PYG{p}{)}\PYG{p}{]}

\PYG{n}{Hn}\PYG{o}{=}\PYG{p}{[}\PYG{l+m+mf}{1.}\PYG{o}{/}\PYG{l+m+mf}{1.}\PYG{p}{]}

\PYG{k}{for} \PYG{n}{k} \PYG{o+ow}{in} \PYG{n+nb}{range}\PYG{p}{(}\PYG{l+m+mi}{2}\PYG{p}{,}\PYG{n}{nMax}\PYG{o}{+}\PYG{l+m+mi}{1}\PYG{p}{)}\PYG{p}{:}
    \PYG{n}{Hn}\PYG{o}{.}\PYG{n}{append}\PYG{p}{(}\PYG{n}{Hn}\PYG{p}{[}\PYG{o}{\PYGZhy{}}\PYG{l+m+mi}{1}\PYG{p}{]}\PYG{o}{+}\PYG{l+m+mf}{1.}\PYG{o}{/}\PYG{n}{k}\PYG{p}{)}
\PYG{c+c1}{\PYGZsh{}Pour créer le DataFrame, il faut lui donner une matrice de la bonne taille}
\PYG{n}{data}\PYG{o}{=}\PYG{p}{[}\PYG{n}{Hn}\PYG{p}{,}\PYG{n}{logN}\PYG{p}{,}\PYG{n}{logNPlus1}\PYG{p}{]}
\PYG{c+c1}{\PYGZsh{}il faut transposer le dataFrame (pas dans le bon sens)}
\PYG{n}{df}\PYG{o}{=}\PYG{n}{pd}\PYG{o}{.}\PYG{n}{DataFrame}\PYG{p}{(}\PYG{n}{data}\PYG{o}{=}\PYG{n}{data}\PYG{p}{)}\PYG{o}{.}\PYG{n}{T}
\PYG{c+c1}{\PYGZsh{} df.describe() : affiche une description du DataFrame}
\PYG{n}{df}\PYG{o}{.}\PYG{n}{columns}\PYG{o}{=}\PYG{p}{[}\PYG{l+s+s2}{\PYGZdq{}}\PYG{l+s+s2}{\PYGZdl{}H\PYGZus{}n\PYGZdl{}}\PYG{l+s+s2}{\PYGZdq{}}\PYG{p}{,}\PYG{l+s+s2}{\PYGZdq{}}\PYG{l+s+s2}{1+ln(n)}\PYG{l+s+s2}{\PYGZdq{}}\PYG{p}{,}\PYG{l+s+s2}{\PYGZdq{}}\PYG{l+s+s2}{ln(n+1)}\PYG{l+s+s2}{\PYGZdq{}}\PYG{p}{]}
\PYG{n}{df}\PYG{o}{.}\PYG{n}{iplot}\PYG{p}{(}\PYG{n}{kind}\PYG{o}{=}\PYG{l+s+s1}{\PYGZsq{}}\PYG{l+s+s1}{scatter}\PYG{l+s+s1}{\PYGZsq{}}\PYG{p}{,}\PYG{n}{title}\PYG{o}{=}\PYG{l+s+s1}{\PYGZsq{}}\PYG{l+s+s1}{Différentes suites}\PYG{l+s+s1}{\PYGZsq{}}\PYG{p}{)}
\end{sphinxVerbatim}

\begin{sphinxVerbatim}[commandchars=\\\{\}]
\PYG{k+kn}{import} \PYG{n+nn}{plotly}\PYG{n+nn}{.}\PYG{n+nn}{io} \PYG{k}{as} \PYG{n+nn}{pio}
\PYG{k+kn}{import} \PYG{n+nn}{plotly}\PYG{n+nn}{.}\PYG{n+nn}{express} \PYG{k}{as} \PYG{n+nn}{px}
\PYG{k+kn}{import} \PYG{n+nn}{plotly}\PYG{n+nn}{.}\PYG{n+nn}{offline} \PYG{k}{as} \PYG{n+nn}{py}

\PYG{n}{df} \PYG{o}{=} \PYG{n}{px}\PYG{o}{.}\PYG{n}{data}\PYG{o}{.}\PYG{n}{iris}\PYG{p}{(}\PYG{p}{)}
\PYG{n}{fig} \PYG{o}{=} \PYG{n}{px}\PYG{o}{.}\PYG{n}{scatter}\PYG{p}{(}\PYG{n}{df}\PYG{p}{,} \PYG{n}{x}\PYG{o}{=}\PYG{l+s+s2}{\PYGZdq{}}\PYG{l+s+s2}{sepal\PYGZus{}width}\PYG{l+s+s2}{\PYGZdq{}}\PYG{p}{,} \PYG{n}{y}\PYG{o}{=}\PYG{l+s+s2}{\PYGZdq{}}\PYG{l+s+s2}{sepal\PYGZus{}length}\PYG{l+s+s2}{\PYGZdq{}}\PYG{p}{,} \PYG{n}{color}\PYG{o}{=}\PYG{l+s+s2}{\PYGZdq{}}\PYG{l+s+s2}{species}\PYG{l+s+s2}{\PYGZdq{}}\PYG{p}{,} \PYG{n}{size}\PYG{o}{=}\PYG{l+s+s2}{\PYGZdq{}}\PYG{l+s+s2}{sepal\PYGZus{}length}\PYG{l+s+s2}{\PYGZdq{}}\PYG{p}{)}
\PYG{n}{fig}
\end{sphinxVerbatim}


\chapter{Markdown Files}
\label{\detokenize{Part1/markdown:markdown-files}}\label{\detokenize{Part1/markdown::doc}}
Whether you write your book’s content in Jupyter Notebooks (\sphinxcode{\sphinxupquote{.ipynb}}) or
in regular markdown files (\sphinxcode{\sphinxupquote{.md}}), you’ll write in the same flavor of markdown
called \sphinxstylestrong{MyST Markdown}.


\section{What is MyST?}
\label{\detokenize{Part1/markdown:what-is-myst}}
MyST stands for “Markedly Structured Text”. It
is a slight variation on a flavor of markdown called “CommonMark” markdown,
with small syntax extensions to allow you to write \sphinxstylestrong{roles} and \sphinxstylestrong{directives}
in the Sphinx ecosystem.


\section{What are roles and directives?}
\label{\detokenize{Part1/markdown:what-are-roles-and-directives}}
Roles and directives are two of the most powerful tools in Jupyter Book. They
are kind of like functions, but written in a markup language. They both
serve a similar purpose, but \sphinxstylestrong{roles are written in one line}, whereas
\sphinxstylestrong{directives span many lines}. They both accept different kinds of inputs,
and what they do with those inputs depends on the specific role or directive
that is being called.


\subsection{Using a directive}
\label{\detokenize{Part1/markdown:using-a-directive}}
At its simplest, you can insert a directive into your book’s content like so:

\begin{sphinxVerbatim}[commandchars=\\\{\}]
```\PYGZob{}mydirectivename\PYGZcb{}
My directive content
```
\end{sphinxVerbatim}

This will only work if a directive with name \sphinxcode{\sphinxupquote{mydirectivename}} already exists
(which it doesn’t). There are many pre\sphinxhyphen{}defined directives associated with
Jupyter Book. For example, to insert a note box into your content, you can
use the following directive:

\begin{sphinxVerbatim}[commandchars=\\\{\}]
```\PYGZob{}note\PYGZcb{}
Here is a note
```
\end{sphinxVerbatim}

This results in:

\begin{sphinxadmonition}{note}{Note:}
Here is a note
\end{sphinxadmonition}

In your built book.

For more information on writing directives, see the
\sphinxhref{https://myst-parser.readthedocs.io/}{MyST documentation}.


\subsection{Using a role}
\label{\detokenize{Part1/markdown:using-a-role}}
Roles are very similar to directives, but they are less\sphinxhyphen{}complex and written
entirely on one line. You can insert a role into your book’s content with
this pattern:

\begin{sphinxVerbatim}[commandchars=\\\{\}]
Some content \PYGZob{}rolename\PYGZcb{}`and here is my role\PYGZsq{}s content!`
\end{sphinxVerbatim}

Again, roles will only work if \sphinxcode{\sphinxupquote{rolename}} is a valid role’s name. For example,
the \sphinxcode{\sphinxupquote{doc}} role can be used to refer to another page in your book. You can
refer directly to another page by its relative path. For example, the
role syntax \sphinxcode{\sphinxupquote{\{doc\}\textasciigrave{}intro\textasciigrave{}}} will result in: \DUrole{xref,std,std-doc}{intro}.

For more information on writing roles, see the
\sphinxhref{https://myst-parser.readthedocs.io/}{MyST documentation}.


\subsection{Adding a citation}
\label{\detokenize{Part1/markdown:adding-a-citation}}
You can also cite references that are stored in a \sphinxcode{\sphinxupquote{bibtex}} file. For example,
the following syntax: \sphinxcode{\sphinxupquote{\{cite\}\textasciigrave{}holdgraf\_evidence\_2014\textasciigrave{}}} will render like
this: \DUrole{bibtex}{{[}holdgraf\_evidence\_2014{]}}.

Moreoever, you can insert a bibliography into your page with this syntax:
The \sphinxcode{\sphinxupquote{\{bibliography\}}} directive must be used for all the \sphinxcode{\sphinxupquote{\{cite\}}} roles to
render properly.
For example, if the references for your book are stored in \sphinxcode{\sphinxupquote{references.bib}},
then the bibliography is inserted with:

\begin{sphinxVerbatim}[commandchars=\\\{\}]
```\PYGZob{}bibliography\PYGZcb{} references.bib
```
\end{sphinxVerbatim}

Resulting in a rendered bibliography that looks like:




\subsection{Executing code in your markdown files}
\label{\detokenize{Part1/markdown:executing-code-in-your-markdown-files}}
If you’d like to include computational content inside these markdown files,
you can use MyST Markdown to define cells that will be executed when your
book is built. Jupyter Book uses \sphinxstyleemphasis{jupytext} to do this.

First, add Jupytext metadata to the file. For example, to add Jupytext metadata
to this markdown page, run this command:

\begin{sphinxVerbatim}[commandchars=\\\{\}]
\PYG{n}{jupyter}\PYG{o}{\PYGZhy{}}\PYG{n}{book} \PYG{n}{myst} \PYG{n}{init} \PYG{n}{markdown}\PYG{o}{.}\PYG{n}{md}
\end{sphinxVerbatim}

Once a markdown file has Jupytext metadata in it, you can add the following
directive to run the code at build time:

\begin{sphinxVerbatim}[commandchars=\\\{\}]
```\PYGZob{}code\PYGZhy{}cell\PYGZcb{}
print(\PYGZdq{}Here is some code to execute\PYGZdq{})
```
\end{sphinxVerbatim}

When your book is built, the contents of any \sphinxcode{\sphinxupquote{\{code\sphinxhyphen{}cell\}}} blocks will be
executed with your default Jupyter kernel, and their outputs will be displayed
in\sphinxhyphen{}line with the rest of your content.

For more information about executing computational content with Jupyter Book,
see \sphinxhref{https://myst-nb.readthedocs.io/}{The MyST\sphinxhyphen{}NB documentation}.







\renewcommand{\indexname}{Index}
\printindex
\end{document}